\section{\consideration}
\paragraph{\kadaiaa}
実験結果より,原画像に近い画像は,緑チャネルだけを抜き出してグレイスケール画像として表示したものである.
また,赤チャネルと青チャネルを入れ替えた画像では,文字や建物の概形を,はっきりとらえることができる.
これらは,\eqref{equ:grayscale}でGreenの割合が一番多い理由と考えられる.
\paragraph{量子化数変換・階調変換・閾値処理}
量子化数を減らすことで,等高線のようなものが見える.
これは,元画像でなだらかな画素値の変化箇所が,量子化数を減らすことでとびとびの値になることで生じる.これを擬似輪郭という.
擬似輪郭は,量子化数を減らすことでより顕著になる.
量子化数が減ると擬似輪郭が生じる.当然,この擬似輪郭は階調反転しても変わらない.
さらに,閾値を調節することで,擬似輪郭を境に明暗がはっきりと分かれた.
\paragraph{画像処理フィルタ}
メディアンフィルタは,ノイズに強いことが分かる.これは中央値を算出することにより,ノイズでない値が中央値となるからである.
ただし,画素値の中央値を抽出するため,平均値を算出する平滑化フィルタに比べて,計算量が多い.
それに対して,平滑化フィルタは,ノイズの画素値を含めた値の平均を算出するので,メディアンフィルタに比べるとノイズに弱い.
Sobelフィルタは,画像の隣接画素間の差分を計算する.よって,差が激しい画素の画素値が大きくなり,白く表示される.
つまり,ノイズに対してもエッジが強調される.
Laplacianフィルタは,画像の2次微分を計算することでエッジの位置を抽出している.Laplacianフィルタは高周波成分を増幅するため,ノイズに対してもエッジが強調される.
\paragraph{背景差分画像・色空間変換}
背景差分画像では,被写体がぼんやりと写っている.この画像に対して閾値処理することで,被写体を強調できることが分かった.
今回の実験では,閾値を手探りで探したため,実験結果の閾値が最適か分からなかった.
また,実験結果より,強調された部分は,光の当たり方や影の変化で白く光る部分もある.
ゆえに,背景差分画像だけでは,被写体の輪郭を正確にとらえることができない.
HSV色空間では,この問題を解決できた.
RGB色空間では,明暗(影や光の状況)でRGB値が変わるのに対して,HSV色空間では,明暗が「明度」というチャネルで保存されている.
HSV色空間での肌色領域抽出は,光の状況や影に左右されなかったため肌色領域をより正確に抽出できたと考えられる.
\begin{wrapfigure}{r}[0mm]{.3\textwidth}
    \centering
    \begin{tikzpicture}
        \draw[latex-latex](-0.5,1.5)node[left]{\rotatebox{90}{\tiny High}}--(3.5,1.5)node[right]{\rotatebox{90}{\tiny High}};
        \draw[latex-latex](1.5,-0.5)node[below]{\tiny High}--(1.5,3.5)node[above]{\tiny High};
        \draw (0,0)rectangle(3,3);
        \coordinate (O) at (1.5,1.5);
        \node[below right] at (O) {\tiny Low};
        \draw[Stealth-Stealth](4,0)--(4,3)node[midway,fill=white]{\rotatebox{90}{\tiny \(y\)方向スペクトル}};
        \draw[Stealth-Stealth](0,-1)--(3,-1)node[midway,fill=white]{\tiny \(x\)方向スペクトル};
        \draw[-Stealth]($(O)+(-0.5cm,0.5cm)$)node[left]{\tiny 直交成分}--(O);
    \end{tikzpicture}
    \caption{周波数スペクトル}
    \label{fig:周波数スペクトル}
    \vspace{-.5cm}
\end{wrapfigure}
\paragraph{2次元フーリエ変換}
2次元フーリエ変換により得られるパワースペクトルは,中央付近に低周波数,その周辺に高周波成分が分布する.
パワースペクトルの輝度は,含まれている空間周波数成分の量を示す.
正弦波縞パワースペクトルの結果より,\figref{fig:周波数スペクトル}の結果と一致している.
縦縞には横方向に明暗が存在し,縦方向の輝度は変わらないので,パワースペクトルは横方向に存在する.
横縞には縦方向の明暗が存在し,横方向の輝度は変わらないので,パワースペクトルは縦方向に存在する.
さらに,縞の数が多い,つまり空間周波数が高いと,パワースペクトルは外側が明るくなることも,実験結果と一致する.
一方,実験結果より,画像座標とパワースペクトル座標は,依存関係にないことがわかる.
これは\eqref{equ:2次元離散フーリエ変換}からも,点\((x,y)\)の座標値\(x,y\)が,点\((u,v)\)の座標値\(u,v\)に影響を与えないことが分かる.
同じ大きさの長方形の位置によってパワースペクトルは,完全一致はしないものの,大きく異ならないことが実験結果から分かる.
また,パワースペクトルの中心は,直流成分である.直流成分は,\eqref{equ:直交成分}より,画像全体の画素値を足し合わせたものである.
パワースペクトルを確認すると,中心が明るくなっている.これは,縞や長方形の直交成分であろう.
\paragraph{高域通過フィルタ}
低周波領域にフィルタをかけて,高周波領域のみを通過させるフィルタを適用すると,縦横方向のエッジの間隔が狭い領域を確認できた.
具体的には,髪の毛や目,帽子や口などを確認できた.
円形フィルタの半径を大きくすると,さらにエッジの間隔が狭い部分のみ確認できた.
具体的に,フィルタ半径\(50\textrm{pixel}\)の画像で確認できた帽子や口は,フィルタ半径\(100\textrm{pixel}\)の画像では確認できなかった.
\section{結論}
今回の実験では,画像フィルタのを用いたノイズの除去や,画像をのエッジを抽出した.
メディアンフィルタは平滑化フィルタに比べてノイズをよく取り除けることが分かった.\par
また,被写体の抽出をするために,画像の背景差分をとる手法と,HSV色空間上での肌色領域を抽出する手法を実験した.
結果より,HSV色空間では「明度」の指標があることからRGB色空間や背景差分を取る手法に比べて,照明の状態に左右されないことが分かった.\par
2次元フーリエ変換では,パワースペクトルと,画像の関係を明らかにした.画像座標と,パワースペクトル座標は関係ないことが明らかになった.
また,高域通過フィルタを適用し,低周波数領域を削除した.逆フーリエ変換により,エッジの感覚が狭い部分のみが抽出できた.
今回の実験では,フィルタにより直行成分まで削除されている.直行成分が画像上でどのような意味を成すか,実験を通した考察は今後の課題としたい.