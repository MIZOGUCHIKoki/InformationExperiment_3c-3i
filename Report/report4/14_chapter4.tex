\chapter{\kadaie}
\section{\purpose}
脳波は以下のように説明されている.
\begin{quote}
    ``神経細胞の間にあるシナプス電位と,後電位などの電極変動の総和を,頭皮上につけた電極を用いて記録したもの.''
    \\\hfill\cite{自己心理学セミナー}
\end{quote}
また,特定の事象に関係する脳波を「事象関連電位(ERP)」と呼ぶ.脳波測定で,ERPを電極で測定する.
BMI(Brain Machine Interfae)とは,脳からの信号を計測し,それを利用する技術を指す\cite{脳波による実用的なBMI研究開発}.
これまで,BMIを用いた家電の操作や,文字入力操作が研究されている\cite{脳波による実用的なBMI研究開発}.
\paragraph{目的}BMIのシステムを体験するために,タイプしたい文字を脳情報から読み取る.ターゲット刺激が表示されたときの脳波と,ターゲット刺激でない刺激が表示されたときの脳波を比較する.
\section{\method}
\paragraph{実験装置}
刺激呈示用装置として,汎用コンピュータと液晶ディスプレイを利用する.
脳波測定器やコンピュータなどの実験装置を\tblref{tbl:実験装置\kadaie}に示す.
\begin{table}[H]
    \caption{実験装置\ (\kadaie)}
    \label{tbl:実験装置\kadaie}
    \begin{tabularx}{\textwidth}{cAR}
        \hline
        \multirow{3}{*}{\bfseries BMI装置}  & 電極       & ドライ電極                                            \\
                                          & 脳波計測機器   & \texttt{g.USBamp}                                \\
                                          & 信号分析機    & \texttt{g.BSanalyze}                             \\
        \hline
        \multirow{6}{*}{\bfseries コンピュータ} & コンピュータ   & Dell Precision M6600 64-bit Operating System     \\
                                          & プロセッサ    & Intel(R) Core(TM) i7-2620M SPC @ 2.70GHz 2.70GHz \\
                                          & メモリ      & 8GB (7.88GB usable)                              \\
                                          & OS       & Windows 7 Ultimate Service Pack 1                \\
                                          & \matlab  & バージョン不明                                          \\
                                          & Simulink & バージョン不明                                          \\
        \hline
    \end{tabularx}
\end{table}
\section{\result}
\section{\consideration}
\section{\conclusion}
