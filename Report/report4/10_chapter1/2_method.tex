\section{\method}
\newcommand{\elt}{\texttt{Eye Link Ⅱ}}
\newcommand{\csv}{\texttt{CSV}}
\paragraph{実験装置}
眼球運動の計測には,\elt(SR Research社)を用いる.刺激呈示用装置として,汎用コンピュータと液晶ディスプレイを利用する.
データ分析するソフトウェアには\matlab を用いる.
\begin{table}[H]
    \caption{実験装置\ (\kadaia)}
    \label{tbl:実験装置\kadaia}
    \begin{tabularx}{\textwidth}{cAR}
        \hline
        {\bfseries 眼球運動計測器}                    & \elt                     & SR Research社                                     \\
        \hline
        \multirow{3}{*}{\bfseries 刺激呈示コンピュータ}  & プロセッサ                    & Intel(R) Core(TM) i7-2600 CPU @ 3.40GHz 3.40 GHz \\
                                               & メモリ                      & 4GB                                              \\
                                               & OS                       & Microsoft Windows 7 Professional Service Pack 1  \\
        \hline
        \multirow{6}{*}{\bfseries データ分析コンピュータ} & コンピュータ                   & MacBook Air 2022 (Apple社)\texttt{MLY13J/A}       \\
                                               & プロセッサ                    & Apple Silicon M2\ \  8コアCPU,8コアGPU               \\
                                               & メモリ                      & 8GB                                              \\
                                               & OS                       & macOS 13.4                                       \\
                                               & \multirow{2}{*}{\matlab} & R2023a - academic use (Update1 9.14.02239454)    \\
                                               &                          & 64-bit (maci64) March 30, 2023                   \\
        \hline
    \end{tabularx}
\end{table}
\paragraph{眼球運動の計測}
顔の印象に関する判断を行うときの,眼球運動の計測を行う.
\elt のアイカメラを装着し,キャリブレーション\footnote{被験者の中心点を,実験機に記憶させる手続き.}およびバリデーション\footnote{キャリブレーションが正常に完了したことを確認する手続き}を行う.次に固定視点を注視し,左右に現れる顔画像のどちらが魅力的かを判断し,キーを押して反応する.この試行20回行う.
\paragraph{実験データの分析}
実験データを\csv ファイルで出力したものでデータの分析を行う.出力された\csv ファイルのIndex を\tblref{tbl:全体CSV意味\kadaia},\tblref{tbl:部分CSV意味\kadaia}に示す.
\begin{table}[H]
    \scriptsize
    \centering
    \caption{\texttt{exp4i\_g0310.csv}}
    \label{tbl:全体CSV意味\kadaia}
    \renewcommand{\arraystretch}{1.3}
    \begin{tabularx}{\textwidth}{|CCCC|}
        \multicolumn{1}{C}{1}           & 2                               & 3     & \multicolumn{1}{C}{4} \\
        \hline
        \multirow{2}{*}{\bfseries 試行回数} & \multirow{2}{*}{\bfseries 選択画像} & 不明    & 不明                    \\
                                        &                                 & 今回未使用 & 今回未使用                 \\
        \hline
    \end{tabularx}
    \caption{\texttt{g0310.asc\_TRIAL\_N.csv}\ \ \((\texttt{N}={1,2,\dots 20})\)}
    \label{tbl:部分CSV意味\kadaia}
    \begin{tabularx}{\textwidth}{|CCCCCC|}
        \multicolumn{1}{C}{1}             & 2                    & 3                                & 4                                        & 5                         & \multicolumn{1}{C}{6} \\
        \hline
        \multirow{2}{*}{\bfseries 時間(ms)} & {\bfseries 測定する目の情報} & \multirow{2}{*}{\bfseries 画面の状態} & \multirow{2}{*}{\bfseries 左右注視位置\(x\)座標} & {\bfseries 左目注視位置\(y\)座標} & {\bfseries 左目瞳孔径}     \\
                                          & 今回未使用                &                                  &                                          & 今回未使用                     & 今回未使用                 \\
        \hline
    \end{tabularx}
\end{table}
\begin{wrapfigure}{r}[0mm]{.35\textwidth}
    \centering
    \begin{framed}
        \scriptsize
        \begin{minipage}[t]{.48\textwidth}
            \begin{itemize}
                \setlength{\leftskip}{-2em}
                \item {\bfseries 選択画像}
                      \begin{itemize}
                          \setlength{\leftskip}{-3em}
                          \item \texttt{100}:右を選択
                          \item \texttt{102}:左を選択
                      \end{itemize}
            \end{itemize}
        \end{minipage}
        \begin{minipage}[t]{.48\textwidth}
            \begin{itemize}
                \setlength{\leftskip}{-2em}
                \item {\bfseries 画面の状態}
                      \begin{itemize}
                          \setlength{\leftskip}{-3em}
                          \item \texttt{0}:画面表示なし
                          \item \texttt{1}:注視点
                          \item \texttt{2}:画像呈示
                      \end{itemize}
            \end{itemize}
        \end{minipage}
    \end{framed}
    \vspace{-1cm}
\end{wrapfigure}
\paragraph{欠損値}
今回の実験機\elt は,時間周波数\(500\textrm{Hz}\)でデータを取得している.
この場合,隣接するサンプル間の時間感覚は\(2\textrm{ms}\)となる.
しかし,眼球の向きが計測できない場合(瞬きなど),その部分のデータは記録できないため欠損値として扱う.
欠損値の検出方法手続きは,以下の通りである.
\begin{enumerate}
    \item
\end{enumerate}
