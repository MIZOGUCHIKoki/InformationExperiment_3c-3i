\chapter{\kadaia}
\section{\purpose}
\paragraph{カスケード現象}
嗜好判断が下される際の眼球運動に注目した研究がある.複数の対象を比較しながら好みのものを選ぶ際には,選好する刺激に対して頻繁に視線を向けること\footnote{選好注視と呼ばれている.}が知られている
ここで次のような実験が行われた.左右に並んだ顔写真を呈示し,「どちらが魅力的か」を判断してもらう.顔画像を呈示してから,判断するまでの眼球運動を分析すると,画像呈示直後は両者の顔画像をほぼ50\%ずつの割合で見比べるが,その後,片方の顔画像を見る確率が次第に増加し,80\%を超えたところで長く見た顔写真を「魅力的だ」と判断して決定する.
この実験中の「片方の顔画像を見る確率が次第に増加する」現象を,視覚のカスケード現象(以下,カスケード現象)と呼ぶ.
Shimojo et al(2003)は,一方の顔画像に対して注視時間が長くなることによりもう一方の刺激を精査する時間が短くなることによって生じると解釈した.\hfill\cite[p.202]{美感},\cite{潜在呈示した情報が選択判断時の視線の動きに与える影響}\par
\paragraph{目的}この実験では,意思決定直前の眼球運動の計測を行う.その結果を\matlab で分析し,カスケード現象が実際に起きているか確認する.
カスケード現象の有無と,理想データと実験データの差異について考察する.