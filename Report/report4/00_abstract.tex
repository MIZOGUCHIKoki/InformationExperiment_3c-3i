\renewcommand{\abstractname}{概要}
\begin{abstract}
    今回の実験では,眼球運動測定実験,行動実験,脳波測定実験をする.
    眼球運動では,顔画像の印象に関する判断についての眼球運動と,対象を自由に見るときの眼球運動を計測する.
    カスケード現象より選好する顔画像を選択する直前の視線の向き,またポスターに対する注視する箇所について明らかになった.
    行動実験では,多数ある妨害刺激中に目標刺激があるか否かを判断し,刺激数と探索時間の関係を確認する.
    結果に対して回帰直線の傾きを求め,刺激の種類によって,傾きの異なることが,統計的仮説検定を用いて明らかとなった.
    脳波測定実験では,BMI装置を用いて脳波を読み取り,指定したターゲット刺激が表示された脳波は,そうでない場合の脳波に比べて電位の振れがより目立った.
\end{abstract}
