\renewcommand{\abstractname}{概要}
\begin{abstract}
    今回の実験では,眼球運動測定実験,行動実験,脳波測定実験をする.
    眼球運動では,顔画像の印象に関する判断についての眼球運動と,対象を自由に見るときの眼球運動を計測する.
    カスケード現象より選好する顔画像を選択する直前の視線の向き,またポスターに対する注視する箇所について明らかになった.
    行動実験では,多数ある妨害刺激中に目標刺激が有るか否かを判断し,刺激数と探索時間の関係を確認する.
    結果に対して回帰直線の傾きを求め,刺激の種類によって回帰直線が異なることが,統計的仮説検定を用いて明らかとなった.
    脳波測定実験では,BMI装置を用いて脳波を読み取り,指定したターゲット刺激が表示された脳波はそうでない場合の脳波に比べて電位の振れが大きいことが分かった.
\end{abstract}
