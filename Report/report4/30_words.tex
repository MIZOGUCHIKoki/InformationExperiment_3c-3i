\chapter{関連語句}
\section{脳情報デコーディング}
脳情報デコーディングとは,脳活動を読み取る技術である.
デコーディングとは,符号列を元の状態に戻す作業を指す言葉である.
たとえば脳が外界から何らかの刺激を受け取っていた場合,脳情報デコーディング技術を用いて,その刺激について第三者が解読可能である.
脳情報デコーディングは,外界の刺激や行動,認知状態と,その時の脳活動との関連性を学習するステップと,
その学習結果をもとに脳活動から外界の刺激や行動,認知状態を予測するステップからなる.
脳情報デコーディングを用いた研究例として,運動情報の読み取りが挙げられる.
ヒト運動野のfMRI信号を用いて,ジャンケンを行う例では,ヒトが「チョキ」を出したときの脳活動を計測し,脳活動パターン解析結果により,限りなくリアルタイムに近い状態でロボットに実験参加者と同じジャンケンをさせることができる.
今回の課題では,第\ref{chap:\kadaie}章の\kadaie が脳情報デコーディングにあたる.\\
\hfill\cite{脳情報デコーディング技術とその応用}
\section{視線入力インタフェース}
視線入力インタフェースとは,視線により,利用者の意思を入力するシステムであり,肢体障がい者のコミュニケーション手段として研究されてきた.
視線入力システムには,いくつか方法があるが,ここでは大きく「眼球運動測定法」を取り上げ,
眼球運動測定法はさらに,眼電図法(EOG),強膜反射法,角膜反射法,瞳孔\ -\ 角膜反射法に分けられている.
ここでは角膜反射法について取り上げよう.角膜反射法とは,角膜上に光の反射点を生じさせ,眼球の画像を撮影した後,
撮影した眼球の画像から,角膜上の光の反射点と瞳孔を識別する.この結果により眼球の方向を算出する.
この技術は,第\ref{chap:\kadaia}章の\elt ,第\ref{chap:\kadaid}章の\tobi に用いられている技術である.\\
\hfill\cite{weko_847_1}
\section{顕著性マップ}
\section{分散分析}