\chapter*{初めに}
\addcontentsline{toc}{chapter}{初めに}
\section*{実験環境}
\addcontentsline{toc}{section}{実験環境}
この実験に用いいる言語はMathworks社の\matlab である.実行環境は以下の通り.
\begin{table}[h]
    \caption{実験環境}
    \label{tbl:実験環境}
    \begin{tabularx}{\textwidth}{AR}
        \hline
        実験機                      & Mac Studio 2022 (Apple社) \\
        プロセッサ                    & Apple Silicon M1 Max     \\
        メモリ                      & 32GB                     \\
        \multirow{2}{*}{\matlab} & R2023a 9.14.02206163     \\
                                 & 64-bit (maci64)          \\
        \hline
    \end{tabularx}
\end{table}
\section*{レポート構成}
\addcontentsline{toc}{section}{レポート構成}
本レポートは,各課題について章を割り当て,各課題ごとの実験方法,実験結果や考察を述べる.\\
\begin{table}[h]
    \begin{minipage}[t]{.48\textwidth}
        \begin{description}
            \item[第\ref{chap:音の工学的特徴}章] 音の工学的特徴
                \begin{enumerate}
                    \item \kadaiaa
                    \item \kadaiab
                    \item \kadaiac
                    \item \kadaiad
                    \item \kadaiae
                \end{enumerate}
            \item[第章] 周波数解析
                \begin{enumerate}
                    \item
                \end{enumerate}
        \end{description}
    \end{minipage}
    \begin{minipage}[t]{.48\textwidth}
        \begin{description}
            \item[第章] 音声合成
                \begin{enumerate}
                    \item
                \end{enumerate}
            \item[第章] >>
                \begin{enumerate}
                    \item
                \end{enumerate}
        \end{description}
    \end{minipage}
\end{table}