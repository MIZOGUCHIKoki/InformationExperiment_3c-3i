\chapter{目的及び背景}\label{chap:音の工学的特徴}
近年,音声技術の目紛しい進化で日常生活で音声技術が身近に感じつつある.
例として合成音声技術で駅の自動放送案内,AIアシスタントの返答読み上げや,音声認識技術でAIアシスタントへの命令や音声文字入機能などが挙げられる.
これらの技術は,音声データに対して処理や演算を施したり,音声として聞き取れるようなデータを構築することが基盤となって成り立っている.\par
本実験では,これらの基盤となる音声データの加工や解析を計算機上で実現し,理論と結果について考察する.\par
具体的な目標を以下に示す.
\begin{enumerate}
    \item 正弦波の生成をプログラミングを用いて作成し,周波数,振幅や初期位相の条件に対して音や正弦波がどのように変化するか実験により確認する.また,白色ガウス雑音や,フーリエ級数展開を用い矩形波を,計算機上で再現する.理論上の矩形波と計算機上での矩形波の違いについて考察する.
    \item フーリエ変換による周波数解析や,特定の周波数のみを通過させるフィルタを作成し,音声データに対して適用させる.
    \item 音声データや周波数解析結果に基づいて音声合成技術を簡易的に実現する.音声データを横軸基準に上下反転させ,「ノイズキャンセリング機能」の簡易的な実現や,周波数解析により特定母音の高調波を取り出し機械的に音声を合成する.合成した音声と合成前の音声の音の違いを観測する.
    \item ???
\end{enumerate}