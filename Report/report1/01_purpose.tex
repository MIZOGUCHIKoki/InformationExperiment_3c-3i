\chapter{目的及び背景}
近年,音声技術の目紛しい進化で日常生活で音声技術が必要不可欠である.
例として合成音声技術で駅の自動放送案内,AIアシスタントの返答読み上げや,音声認識技術でAIアシスタントへの命令や音声文字入機能などが挙げられる.
これらの技術は,音声データに対して処理や演算を施したり,音声として聞き取れるようなデータを構築することが基盤となって成り立っている.\par
本実験では,これらの基盤となる音声データの加工や解析を計算機上で実現し,理論と結果について考察する.\par
具体的な目的を以下に示す.
\begin{enumerate}
	\item 正弦波の生成をプログラミングを用いて作成し,周波数,振幅や初期位相の条件に対して音や正弦波がどのように変化するか実験により確認する.また,白色ガウス雑音や,フーリエ級数展開を用い矩形波を,計算機上で再現する.理論上の矩形波と計算機上での矩形波の違いについて考察する.
	\item フーリエ変換による周波数解析や,特定の周波数のみを通過させるフィルタを作成し,音声データに対して適用させる.
	\item 音声データや周波数解析結果に基づいて音声合成技術を簡易的に実現する.音声データを横軸基準に上下反転させ,「ノイズキャンセリング機能」の簡易的な実現や,周波数解析により特定母音の高調波を取り出し機械的に音声を合成する.合成した音声と合成前の音声の音の違いを観測する.
	\item ???
\end{enumerate}

\begin{enumerate}
	\item 我々は音の「高い」「低い」をどのようにして認識しているのだろうか.音が高いまたは低いと感じるためには何かと比較するはずだ.その比較の指標は何だろうか.正弦波の生成をプログラミングを用いて作成する.そして周波数の変化に対して正弦波グラフおよび音の違いを実験を通して確認し,考察する.
	\item 音の大小は何によって決まるのだろうか.音の大小に関わる波の振幅や波を構成する上で重要な初期位相を変化させ,変化の前後での音の違いを聞き取り,人間の耳に初期位相の変化や振幅の変化がどのように感じるか実験を通して考察する.
	\item ティンパニやギターのチューニングを行うとき,音叉やチューナーから音を出して,異なる互いに周波数であれば気づきチューニングする.計算機で意図的にうなりを発生させ,それぞれ異なる周波数が異なる周波数が周波数の違いによるうなりの発生やその原因を数学的観点から考察する.
	\item 矩形波は周期関数だが見た目は正弦波ではない.これをプログラム上でどのように再現するのか.フーリエ級数展開を用いた矩形波の描画やフーリエ級数展開のプログラム上で再現し,理想的な矩形波との比較を行う.
	\item 我々が「雑音」と呼ぶものは,どのような波形をしているのだろうか.今回は雑音の中でも特殊な「白色雑音」,「ガウス雑音」を実装その雑音を聴音する.白色雑音とガウス雑音の定義は以下.
	      \begin{description}
		      \item[白色雑音] すべての周波数成分を均等に含む,パワースペクトルが一定である不規則性の非常に強い波のことである.\cite{whitenoise}
		      \item[ガウス雑音] \textit{Gaussian noise represents statistical noise having the probability density function equal to that of the normal distribution.}
	      \end{description}
\end{enumerate}