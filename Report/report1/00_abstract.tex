\renewcommand{\abstractname}{概要}
\begin{abstract}
    音声技術は日常生活で様々な場面で利用されている.例えば駅の放送案内やAIアシスタントの合成音声,イヤフォンのノイズキャンセリング機能,音声をテキストに変換するソフトウェアなど挙げればきりがない.これらの技術はどのようにして実現できているのだろうか.そのために音の工学的特徴を捉える.\par
    また,音は様々な周期や振幅を持つ正弦波の重ね合わせで再現されていることを踏まえて,周波数成分を取り出して解析する周波数解析を行い,周波数解析の結果を用いて周波数を編集し波形を再構成する.\par
    最後に人間の聴覚特性を再現する.人間の聴覚特性である音脈分凝,連続聴効果またミッシングファンダメンタル波刺激を起こす音波を計算機上で作成する.
\end{abstract}