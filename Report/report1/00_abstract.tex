\renewcommand{\abstractname}{概要}
\begin{abstract}
    音を計算機上で表現するために,行列を用いる.基本となる正弦波の振幅や位相と,音の関係について実験する.さらに,周波数の異なる音の加算合成で起こるうなり現象,統計的な雑音である白色ガウス雑音の実装も行う.\par
    また,音は,さまざまな周期や振幅を持つ正弦波の重ね合わせで再現されていることを踏まえて,周波数成分を取り出して解析する周波数解析し,周波数解析の結果を用いて周波数を編集し波形を再構成する.\par
    最後にヒトの聴覚特性を再現する.ヒトの聴覚特性である音脈分凝,連続聴効果またミッシングファンダメンタル波刺激を起こす音波を計算機上で作成し,その効果の再現を行う.効果の起こる条件や考えられる原因を考察する.
\end{abstract}