\renewcommand{\abstractname}{概要}
\begin{abstract}
    音声技術は日常生活で様々な場面で利用されている.例えば駅の放送案内やAIアシスタントの合成音声,イヤフォンのノイズキャンセリング機能,音声をテキストに変換するソフトウェアなど挙げればきりがない.これらの技術はどのようにして実現できているのだろうか.それを知るためには音の工学的特徴を捉える必要がある.\par
    本実験では音の工学的特徴を捉えた上で,音声信号に対して加工や演算を施し,日常に溢れている音声技術を簡易的に再現する.さらにそれらがどのような原理で成り立っているのか考察する.\par
    音はマイクロフォンから得られる電圧の連続データであるが,それをコンピュータに適用させるために非連続化「離散化」を行う.離散化したデータに対して演算して波形を反転させる.また,音は様々な周期や振幅を持つ正弦波の重ね合わせで再現されていることを理解した上で,周波数成分を取り出して解析する周波数解析を行う.解析結果に対して周波数成分を取り出し音声合成を行う.\par
    この実験を通して,日常に溢れている音声技術の一部を再現でき,仕組みについて考察した.実際の音声データの演算は単純なプログラムでは計算に時間を要するので,音声データ演算の高速化についても今後研究したい.
\end{abstract}