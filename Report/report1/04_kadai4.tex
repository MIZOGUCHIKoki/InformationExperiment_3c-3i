\chapter{聴覚と音声信号}
\section{\kadaida}\label{sec:\kadaida}
\purpose
\paragraph{音脈分凝とは}
例として,音列\ \fbox{C4\ G4\ C4\ G4\ C4\ G4\ \dots}\ を挙げる.この音列を長く聴いた場合,テンポが遅いと音列は「旋律」として聞こえるが,テンポを早くするについれて\ \fbox{C4\ C4\ C4\ \dots} \ \fbox{G4\ G4\ G4\ \dots}と,音列が分離して聞こえる.
この現象を音脈分凝と呼ぶ.また,テンポを一定にした場合,2つの音程差(今回は完全5度)が大きいほど,音脈分凝が生じやすい.\cite[p.182]{音脈分凝}
\paragraph{実験}
今回の実験では,テンポを一定にした場合の,周波数による音脈分凝の生じ方について調査する.
音脈分凝が生じる波形を作成し,音脈分凝が生じる周波数組み合わせと,音脈分凝が生じにくい組み合わせを実験により発見する.
\method
サンプリング周波数を\(16\textrm{kHz}\),1音を\(0.2\)秒に設定し,音Aと音Bを連結させた音Cを10回繰り返す.
正弦波の行列を連結するために,\texttt{repmat}関数を用いる.実験する周波数の組み合わせを\tblref{tbl:音脈分凝_実験結果}に示す.
\scall\sref{src:04_01}
\result
聴音確認による結果を\tblref{tbl:音脈分凝_実験結果}に示す.
\begin{table}[h]
    \centering
    \caption{音脈分凝\ 実験結果}
    \label{tbl:音脈分凝_実験結果}
    \begin{tabularx}{\textwidth}{cccR}
            & 周波数\(f_A\) & 周波数\(f_B\) & \multicolumn{1}{c}{実験結果}                                    \\
        \hline
        実験1 & \(1000\)   & \(990\)    & 音脈分凝は生じず,滑らかな音に聞こえた.                                        \\
        実験2 & \(1000\)   & \(800\)    & 音脈分凝は生じず,別音の組み合わせによる旋律に聞こえた.                                \\
        実験3 & \(1000\)   & \(200\)    & 音脈分凝が生じ,\(1000\textrm{Hz}\),\(200\textrm{Hz}\)の2音が分離して聞こえた. \\
        \hline
    \end{tabularx}
\end{table}
\consideration
結果より,周波数組み合わせの差\(f_D=\big|f_A-f_B\big|\)の大きいほど,音脈分凝が生じやすいこと,\(f_D=10\)程度であれば,音の切れ目すら聞こえないことが分かった.
今回の実験では,3つの組み合わせのみ実験したため,音脈分凝が生じる具体的な周波数差\(f_D\)は分からなかった.\par
音脈分凝の生じやすさは,個人の近くに左右されるので正確な実験方法,テンポによる音脈分凝の生じやすさについても,今後の課題として調査したい.
\section{\kadaidb}\label{sec:\kadaidb}