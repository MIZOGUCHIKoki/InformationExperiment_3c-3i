\chapter{結論}
今回の実験を通して,正弦波の操作と実際の音波を比較した.振幅の変化は音量の変化として聞こえ,初期位相の変化による音の変化は知覚できなかった.さらに,フーリエ級数展開を用いた周期関数や,統計的な雑音も音波や波形で表現し,考察した.\par
波形の周期に対して振幅をとる周波数解析を用いて,特定周波数鑿を通過させるフィルタの作成,適用や,音声を合成した.音声合成は,特定周波数のみを抽出して再合成した.原音とは程遠いまでも,母音の発音は聞き取れた.また,正弦波では知覚できなかったが,矩形波やノコギリ波など高調波を多く含む波形では,初期位相の変化による音の変化を知覚できた.\par
最後に,人間の聴覚特性を再現するための音波を作成した.音脈分凝では周波数の組み合わせによって生じやすさが変わること,連続聴効果ではノイズの周波数にない音に対しては生じにくこと,ミッシングファンダメンタル波刺激では高調波成分のみでも基本周波数が聞き取れることを確認した.