\chapter{関連語句}
\section{FIRフィルタ,IIRフィルタ}
FIR,IIRはデジタルフィルタである.
\paragraph{FIRフィルタ}
FIR(Finite Impluse Response)フィルタは,「インパルス応答が有限長である」という意味である\cite[p.92]{音声音響インタフェース実践}.
インパルス応答とは,あるシステムにインパルスと呼ぶ短い信号を入力したときのシステムの出力を指す.
インパルス応答\(h(k)\),入力データ\(x(n)\)に対して,FIRフィルタの出力\(y(n)\)は,\eqref{equ:FIR}である.
\begin{align}
    y(n) & =\sum_{k=0}^{N-1}x(n-k)\cdot h(k)\label{equ:FIR}
\end{align}
FIRでは,その名の通り,「有限」範囲での和になるので,\(N\)は有限である.
位相歪\footnote{周波数によって遅延時間が異なること.}が許容されない場合や,安定性が求められるときにはFIRが最適である.
しかし,FIRはIIRに比べて演算回数が増え,回路設計により必要メモリ数が増えるというデメリットもある.
\paragraph{IIRフィルタ}
IIR(Infinite Impluse Response)フィルタは,「インパルス応答が無限長である」という意味である\cite[p.92]{音声音響インタフェース実践}.
\begin{align}
    y(n) & =-\sum_{k=0}^{N}a_k\cdot y(n-k)+\sum_{k=1}^{M}b_k\cdot x(n-k)\label{equ:IIR}
\end{align}
IIRはFIRに比べてメモリ消費量を抑えられ,高速で演算量も抑えられる.また,周波数特性の滑らかさで決まる,インパルス応答の長が長いとFIRによる演算量が膨大になる.このときIIRを用いる.
\paragraph{関連}
振幅スペクトルに対して,フィルタ処理を行った\ref{sec:\kadaibb}章,\ref{sec:\kadaidb}章に関連する.両方とも有限範囲に対するフィルタを適用したので,FIRを適用したと言える.
\section{ノイズキャンセリング}
ノイズキャンセリングは,「雑音除去」を表す.大きく2つあり,イヤフォンやヘッドフォンなどの音声出力装置についているものと,音声入力装置(マイク)についているものである.
どちらも,外部雑音を除去するために用いられる.\par
\paragraph{音声出力装置}イヤフォンやヘッドフォンに搭載されているノイズキャンセリングでは,イヤフォン外の雑音を一度取り込み,それに対して\ref{sec:\kadaicb}章で扱ったように,各データと\(-1\)の積を取り,その波をイヤフォンから出力することで,雑音を打ち消せる.\par
\paragraph{音声入力装置}Apple社のiOS 15以降では,「声を分離」という機能がある.これは,入力された音声から雑音を取り除く技術である.これは音声入力装置のものとは異なり,雑音を機械学習でとらえ,除去する技術である.
\section{聴覚におけるマスキング}
対象とする音(マスキ)の聴き取りが,別の音(マスカ)の存在によって妨害される現象.
マスキングには2種類あり,同時マスキング,継時マスキングがある.
\paragraph{同時マスキング}同時マスキングは,周波数軸上のマスキング効果.マスキとマスカが同時に提示されたとき,マスカに近いマスキの周波数が聴き取りにくく現象\cite{マスキングと騒音対策}.
\figref{fig:同時マスキング}中の着色部は,不可聴音の領域である.同時マスキングの特徴として,高音が低音よりも音量が大きい場合,高音は低音をマスクしにくいのに対して,低音が高音よりもマスクしやすいことがあげられる.
\paragraph{継時マスキング}継時マスキングは,時間軸上のマスキング効果.マスカを提示した直前や直後のマスキが聴き取りにくくなる現象.
\figref{fig:継時マスキング}中の着色部は,不可聴音の領域である.マスカ提示時刻より前のマスキングを逆行マスキング,マスカ提示時刻より後のマスキングを順行マスキングと呼ぶ.
\begin{figure}[h]
    \centering
    \begin{minipage}[b]{.48\textwidth}
        \centering
        \begin{tikzpicture}
            \fill[blue,opacity=.1](0,0)to[out=0,in=-120](2,2)--(2,2)to[out=-60,in=180](4,0)--cycle;
            \draw[thin](0,0)to[out=0,in=-120](2,2);
            \draw[thin](2,2)to[out=-60,in=180](4,0);
            \draw[-latex](0,0)--(0,3)node[midway,left]{\tiny 音圧レベル};
            \draw[-latex](0,0)--(4,0)node[right]{\tiny 周波数};
            \draw[thick](2,0)node[below]{\tiny マスカ周波数}--(2,2)node[above]{\tiny マスカ};
            \node[fill=blue,opacity=.1,text opacity=1,text centered]at(3,2.6){\tiny 同時マスキング領域};
        \end{tikzpicture}
        \caption{同時マスキング}
        \label{fig:同時マスキング}
    \end{minipage}
    \begin{minipage}[b]{.48\textwidth}
        \centering
        \begin{tikzpicture}
            \fill[red,opacity=.1](1,0)to[out=20,in=-100](2,2)--(2,0)--cycle;
            \fill[green,opacity=.1](2,2)to[out=-60,in=180](4,0)--(2,0)--cycle;
            \draw[thin](1,0)to[out=20,in=-100](2,2);
            \draw[thin](2,2)to[out=-60,in=180](4,0);
            \draw[-latex](0,0)--(0,3)node[midway,left]{\tiny 音圧レベル};
            \draw[-latex](0,0)--(4,0)node[right]{\tiny 時間};
            \draw[thick](2,0)node[below]{\tiny マスカ提示時刻}--(2,2)node[above]{\tiny マスカ};
            \node[fill=red,opacity=.1,text opacity=1]at(1,2.6)(G){\tiny 逆行マスキング領域};
            \node[fill=green,opacity=.1,text opacity=1]at(3,2.6)(J){\tiny 順行マスキング領域};
        \end{tikzpicture}
        \caption{継時マスキング}
        \label{fig:継時マスキング}
    \end{minipage}
\end{figure}
\paragraph{関連}
逆行マスキングに着目する.逆行マスキングは,つまり何らかの刺激(マスカ)の直前の刺激を知覚できない現象である.これは,連続聴効果と関連付けることができる.(\ref{sec:\kadaidb})
逆行マスキングの効果による,日常生活で聞こえない部分を,連続聴効果で補っていると考えられる.
\section{音声認識}
音声認識は,機械が音声をテキストに変換するために用いられている技術である.
音声認識は,大きく2つの処理を経る.入力音声に対して,音響モデルを用いて音響分析を行った後,言語モデルを用いて言語化する.
\begin{enumerate}
    \item \textbf{音響分析}\\
          AIが扱いやすいデータにするための作業を音響分析という.背景音やノイズの除去を行う.
    \item \textbf{音響モデル}\\
          音声分析によって抽出したデータを,AIが過去のデータをもとに音素を抽出する.
          たとえば,\ref{sec:\kadaicc}章の\figref{fig:振幅スペクトルの絶対値_a}と\figref{fig:振幅スペクトルの絶対値_i}比べると,母音によって含まれる高調波の周波数が異なる.(\ref{sec:\kadaicc}章)この違いをもとに音素\footnote{音素とは,音韻の最小単位.例えば,「カ」の音素は\texttt{k}と\texttt{a}である.}を抽出する.
    \item \textbf{発音辞書}\\
          抽出した音素を,単語に構成する過程.
    \item \textbf{言語モデル}\\
          言葉を,単語をもとに生成する.言語モデルは,単語の出現確率でモデル化したもの.
\end{enumerate}