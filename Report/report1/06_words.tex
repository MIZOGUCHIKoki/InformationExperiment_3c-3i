\chapter{関連語句}
\section{FIRフィルタ,IIRフィルタ}
FIR,IIRはデジタルフィルタである.
\paragraph{FIRフィルタ}
FIR(Finite Impluse Response)フィルタは,「インパルス応答が有限長である」という意味である\cite[p.92]{音声音響インタフェース実践}.
インパルス応答とは,あるシステムにインパルスと呼ぶ非常に短い信号を入力したときのシステムの出力を指す.
インパルス応答\(h(k)\),入力データ\(x(n)\)に対して,FIRフィルタの出力\(y(n)\)は,\eqref{equ:FIR}である.
\begin{align}
    y(n) & =\sum_{k=0}^{N-1}x(n-k)\cdot h(k)\label{equ:FIR}
\end{align}
FIRでは,その名の通り,「有限」範囲での和になるので,\(N\)は有限である.
位相歪\footnote{周波数によって遅延時間が異なること.}が許容されない場合や,安定性が求められるときにはFIRが最適である.
しかし,FIRはIIRに比べて演算回数が増え,回路設計により必要メモリ数が増えるというデメリットもある.
\paragraph{IIRフィルタ}
IIR(Infinite Impluse Response)フィルタは,「インパルス応答が無限長である」という意味である\cite[p.92]{音声音響インタフェース実践}.
\begin{align}
    y(n) & =-\sum_{k=0}^{N}a_k\cdot y(n-k)+\sum_{k=1}^{M}b_k\cdot x(n-k)\label{equ:IIR}
\end{align}
IIRはFIRに比べてメモリ消費量を抑えられ,高速で演算量も抑えられる.また,周波数特性の滑らかさで決まる,インパルス応答の長が長いとFIRによる演算量が膨大になる.このときIIRを用いる.
\paragraph{関連}
振幅スペクトルに対して,フィルタ処理を行った\ref{sec:\kadaibb}章,\ref{sec:\kadaidb}章に関連する.両方とも有限範囲に対するフィルタを適用したので,FIRを適用したと言える.

\section{ノイズキャンセリング}
\section{聴覚におけるマスキング}