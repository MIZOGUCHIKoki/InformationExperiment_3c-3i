\section{結論}
この実験を通して,色覚特性を根拠とした,色覚障害者の見え方の違いについて明らかになった.
私自身は一般色覚を持っており,例えば私は「赤色」を\rulebox{{red}}{red}と認識できる.
しかし\rulebox{{red}}{red}を誰しも認識できる前提で,重要な事項(避難ルートや警報)を伝える看板や掲示板を設計すると,色弱者は文字や意味を読み取れずに命を落としかねない.
カラーユニバーサルデザインは,色弱者は一般色覚を持つ人と平等なサービスを享受するために必要なデザインであることが明らかになった.\par
このレポートでは,カラーユニバーサルデザインに配慮した設計になっている.色を示すときに用いたカラーボックス(\rulebox{{orange}}{orange})や,線グラフの形状がその例である.