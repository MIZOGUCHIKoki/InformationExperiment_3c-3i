\section{関連語句}
\subsection{DOM}
DOMとは,Document Object Model の略称で,プログラムから構造化された文書を扱うためのモデルである.
DOMは,オブジェクトを操作するAPIを提供している.JavaScriptなどのいろいろなプログラミング言語から利用できる.
DOMは,データを木構造で扱う.この構造の構成要素をノード(Node)という.
構造化された文書の例としてXMLをあげる.
例として,以下のようなXML文書に対して,DOM tree で表現する.
\begin{lstlisting}[caption={XML文書},language={}]
<?xml version="1.0" encoding="utf-8"?>
<a version="2.0">
  <b>
    <c>
      <d>KUT</d>
      <f>kut.com</f>
    </c>
  </b>
</a>
\end{lstlisting}
\begin{figure}[H]
  \centering
  \begin{lstlisting}[language={}]
|-Document
  `- Element: a vertion="2.0"
     `- Element: b
        `-Element: c
          |- Element: d
          |  `- Text: KUT
          `- Element: f   
             `- Text: kut.com
            \end{lstlisting}
  \caption{DOM tree}
\end{figure}
Documentは文書全体を表すノード,Elementは要素を表すノード,Textはテキストを表すノードである.
ここで,Textも1つのノードとなることに注意したい.
\\\hfill\cite{XML文書とDOM}
\subsection{WebGL}
OpenGL(Open Graphics Library)とは,3次元グラフィックスライブラリである.これを用いることで,高品質な仮想区間を表現できる.構成要素として,3次元立体や2次元画像が挙げられる\cite{OpenGL入門}.
WebGLとは,ブラウザ上で3次元グラフィックを実現する技術である.
HTML5の\texttt{Canvas}要素に対して,ブラウザでJavaScriptなどを使用してOpenGLで描画する.
WebGLは,ブラウザだけで,コンテンツの3次元描画が可能になり,アプリケーションのインストールやバージョンアップは必要なく,OpenGLを用いるため,Web上でネイティブアプリケーションと同等のパフォーマンスを再現できる.
\\\hfill\cite{webglみずほ}
\subsection{three.js}
three.jsは,JavaScriptベースのWebGLエンジンである.グラフィックスを駆使したアプリケーションをブラウザから直接実行できる.
three.jsライブラリは,ブラウザ上での3D描画するための機能とAPIを提供している.
WebGLとthree.jsは,APIかライブラリであるかの違いである.
WebGLはブラウザ上3DグラフィックスをレンダリングするためのJavaScript API であるのに対し,three.jsは,WebGL上で3Dグラフィックスを作成するためのライブラリである.
three.jsは,物理演算や光源の位置により物体の光り方を変更できたりなど,数学やプログラミングの知識がなくとも簡便に3Dグラフィックスを作成できる.
\\\hfill\cite{three.js}