\section{\consideration}
講義一覧,折れ線グラフ,円グラフのコンテンツを,すべての色覚で内容を判別できるように配色した.
折れ線グラフについては,マークの形状を商品別に変更することで,商品を区別した.
マークは線と同化し,一部見にくいので,線の形状やマークの色を変更することで,さらに見やすいデザインになる.
また,円グラフは,色の変更だけでなく,項目間の感覚を開けることで,色の境界に対して見やすさを保証できる.
\paragraph{色覚}
一般色覚(C型色覚)の人は,L(赤)錐体,M(緑)錐体,S(青)錐体の3種類を持っており,すべて正常に働いている.
色弱者(色の配慮が不十分な社会における弱者)は,いずれかの錐体がない,またはいずれかの錐体が不十分な働きである.
たとえば,M錐体がない,または不十分な働きであるD型色覚では,\figref{fig:D型色覚の周波数に対する感度}のように2つの周波数領域で混同が生じる.
\begin{figure}[H]
    \centering
    \begin{tikzpicture}
        \draw[thick,-latex](-.3,0)--(6.5,0)node[midway,below=.4cm]{\small 光の波長(nm)};
        \draw[thick,-latex](-.3,0)--(-.3,3)node[midway,left]{\small \rotatebox{90}{光に対する感度}};
        \foreach \u \v in {0/400,1/450,2/500,3/550,4/600,5/650,6/700}
        \draw(\u,-0.1)node[below]{\scriptsize \v}--(\u,0);
        \foreach \u in {0,0.2,...,6}
        \draw(\u,0)--(\u,0.1);
        \foreach \u in {0.3,0.6,...,3.0}
        \draw(-.3,\u)--(-.2,\u);
        \draw[very thick,blue,dotted]plot[smooth,tension=0.7]coordinates{(0,1.8)(0.8,2.7)(2.8,0.6)(4.8,0)};
        \draw[very thick,gray!80,dashed]plot[smooth,tension=0.7]coordinates{(0,0.6)(0.4,1.2)(0.8,1.5)(2.8,2.7)(5.6,0.3)};
        \draw[very thick,red,densely dash dot dot]plot[smooth,tension=0.7]coordinates{(0,0.6)(0.4,1.2)(1,1.45)(3.2,2.8)(6,0.6)};
        \draw(0.4,0)--(0.4,3);
        \draw(0.8,0)--(0.8,3);
        \draw[latex-latex](0.4,0.6)--(0.8,0.6)node[above right]{\tiny 混同};
        \draw(3.8,0)--(3.8,3);
        \draw(2.8,0)--(2.8,3);
        \draw[latex-latex](2.8,0.6)--(3.8,0.6)node[midway,above]{\tiny 混同};
    \end{tikzpicture}
    \begin{tikzpicture}
        \draw[very thick,blue,dotted](0,0)--(2,0)node[right,color=black]{青に敏感な視細胞(S錐体)};
        \draw[very thick,gray!80,dashed](0,-.5)--(2,-.5)node[right,color=black]{機能しない緑に敏感な視細胞(M錐体)};
        \draw[very thick,red,densely dash dot dot](0,-1)--(2,-1)node[right,color=black]{赤に敏感な視細胞(L錐体)};
    \end{tikzpicture}
    \caption{D型色覚の周波数に対する感度\cite{色覚が変化するとどのように色が見えるのか2}}
    \label{fig:D型色覚の周波数に対する感度}
\end{figure}
C型色覚の場合は,L錐体,M錐体,S錐体それぞれからの入力に対して比を取り,色を弁別している.
それに対して,D型色覚の場合は,L錐体とS錐体の入力の比で色を弁別しているため,L錐体とS錐体の比が同じ箇所は,色の区別がつかないと考えられる.
この混同部分を避けるために,L錐体とS錐体の感度比が異なる部分を採用する必要があるだろう.
\\\hfill\cite{色覚が変化するとどのように色が見えるのか}