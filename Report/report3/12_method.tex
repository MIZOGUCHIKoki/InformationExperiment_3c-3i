\section{\method}
\subsection{装置と用語解説}
\paragraph{実験に用いる装置}この実験にはMathWorks\raisebox{2mm}{\tiny\textregistered}社の\matlab を用いて,\tblref{tbl:実験環境}の環境下で実験する.
\begin{table}[H]
    \caption{実験環境}
    \label{tbl:実験環境}
    \begin{tabularx}{\columnwidth}{cR}
        \hline
        \multirow{2}{*}{実験機}   & MacBook Air 2022 (Apple社)   \\
                               & 型番:\texttt{MLY13J/A}        \\
        \hline
        \multirow{2}{*}{プロセッサ} & Apple Silicon M2            \\
                               & 8コアCPU,8コアGPU               \\
        \hline
        メモリ                    & 8GB                         \\
        \hline
        ブラウザ                   & Safari\ バージョン16.5           \\
        \hline
        ImageJ                 & 1.53k\ Java 13.0.6 (64-bit) \\
        \hline
    \end{tabularx}
\end{table}
\paragraph{HTML}
Hyper Text Markup Languageの略で,Webページを作成するためのマークアップ言語.Webページ上のテキストデータに「タグ」を与えて,文字の大きさ,色やフォントを変更する.
\paragraph{CSS}
Cascading Style Sheetsの略で,Webページのスタイルを指定するための言語である.CSSの適用により,ホームページの文字や背景などがタグを利用して統一される.
\paragraph{JavaScript}
ブラウザ上で動作するアプリケーションを記述するための言語である.現在のWebアプリケーション上でデファクトスタンダード言語である\cite[p.68]{CGとゲームの技術}.
\subsection{JavaScriptの初歩}
\begin{itemize}
    \item \textbf{\texttt{A.indexOf("Strings")}}\\
          JavaScriptには,\texttt{A.indexOf("Strings")}で,\texttt{Strings}が\texttt{A}に含まれているかどうか出力する関数がある.
    \item \textbf{ユーザエージェント(UA)}\\
          ユーザエージェント(UA)とは,利用者のブラウザとOSを指す.
          この実験で利用するUAは,macOSのSafariである.
          SafariにはUAを変更する機能がある.Chrome,Firefox,MicrosoftEdge,Safari 16.4は,SafariのUA切り替え機能を用いて実験する.
\end{itemize}
\begin{enumerate}
    \item \textbf{ブラウザ判定}\\
          \texttt{Navigator object}オブジェクトの\texttt{userAgent}プロパティを用いて,利用ブラウザの判定を行う.
          \texttt{A}に\texttt{Strings}が含まれていない場合,戻り値は\texttt{-1}である.この実験では,ブラウザがFirefoxである場合は\ \texttt{this browser is Firefox}\ ,そうでない場合は\ \texttt{this browser is not Firefox}\ と表示するWebページを作成する.
    \item \textbf{ブラウザによるCSSの切り替え}\\
          ブラウザ判定を用いて,ブラウザによってCSSを変更するJavaScriptを記述する.CSSを設定する\texttt{link}タグに\texttt{id}を指定し,\texttt{document.getElementById}関数で\texttt{link}タグをインスタンス化する.
          ブラウザ判定結果により,このインスタンスを利用してCSSを変更する.この実験では,CSSで背景色のみを変更する.
          \begin{tabularx}{.9\columnwidth}{llR}
              \\
              \multicolumn{1}{c}{UA} & \multicolumn{1}{c}{背景色}  & \multicolumn{1}{c}{CSS名} \\
              \hline
              Firefox                & \rulebox{orange}{orange} & \texttt{firefox.css}     \\
              Chrome                 & \rulebox{blue}{blue}     & \texttt{chrome.css}      \\
              Other                  & \rulebox{gray}{gray}     & \texttt{default.css}     \\
              \hline                                                                       \\
          \end{tabularx}
          ここで,MicrosoftEdgeは,UAに\texttt{"Chrome"}文字列と\texttt{"Edge"}文字列を持つ.正確にUA\texttt{Chrome}を判別するには,条件式に,「\texttt{"Chrome"}を含み\texttt{"Edge"}を除く」処理を記述する必要がある.
    \item \textbf{時刻によるCSSの切り替え}\\
          JavaScriptで現在時刻を取得するには,\texttt{Date}をインスタンス化する必要がある.\texttt{Date}内の\texttt{getSeconds}関数を呼び出すことで,現在時刻の「秒」を得られる.
          この実験では,現在時刻\footnote{正確には,HTMLを読み込んだ時刻.}の秒数を\(n\)とすると,以下の条件でCSSを変更する.
          \begin{equation*}
              \textrm{背景色}=
              \begin{cases}
                  \texttt{firefox.css} & (0\leq n<20)  \\
                  \texttt{chrome.css}  & (20\leq n<40) \\
                  \texttt{default.css} & (40\leq n<60)
              \end{cases}
          \end{equation*}
    \item \textbf{マウスイベントの取得}\\
          マウスイベントを取得するには,
\end{enumerate}
\paragraph{講義一覧の作成}
HTMLの\texttt{table}タグを用いて,講義一覧を作成する.各講義の「専門基礎科目」,「専門発展科目」,「専門領域科目」の背景色と,履修登録必須科目の文字色を,CSSで設定する.
