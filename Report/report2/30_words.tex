\chapter{関連語句}
\section{ガボールフィルタ}
ガボールフィルタとは,画像を周波数領域でテクスチャ解析\footnote{テクスチャ解析とは,粗い,滑らか,でこぼこなどの直感的な材質を,画素強度の空間的な変動の関数として定量化する試み\cite{テクスチャ解析}.}する方法のひとつ.
テクスチャ解析する方法として,フーリエ変換がある.フーリエ変換では画像をいくつかのブロックに区切る必要があり,対象の境界も同時に検出するという問題に対して,結果が粗くなる.
周波数の不正確さと,位置の不正確さとの積には下限があり,それを最小にするのは「ガボール関数」である\cite[p.144]{認知心理学辞典}.\par
ガボール関数は,サルの一次視覚野にある単純細胞の受容野特性によく似ており,ガボール関数の正弦波や余弦波の傾きを変化させることで.この受容野特性をよく再現できる\cite[p.144]{認知心理学辞典}.
\section{固有顔法}
\section{オプティカルフロー}
\section{ストラクチャフロムモーション(SfM: Structure from Motion)}