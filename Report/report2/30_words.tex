\chapter{関連語句}
\section{ガボールフィルタ}
ガボールフィルタとは,画像を周波数領域でテクスチャ解析\footnote{テクスチャ解析とは,粗い,滑らか,でこぼこなどの直感的な材質を,画素強度の空間的な変動の関数として定量化する試み\cite{テクスチャ解析}.}する方法のひとつ.
テクスチャ解析する方法として,フーリエ変換がある.フーリエ変換では画像をいくつかのブロックに区切る必要があり,対象の境界も同時に検出するという問題に対して,結果が粗くなる.
周波数の不正確さと,位置の不正確さとの積には下限があり,それを最小にするのは「ガボール関数」である.\par
ガボール関数は,サルの一次視覚野にある単純細胞の受容野特性によく似ており,ガボール関数の正弦波や余弦波の傾きを変化させることで.この受容野特性をよく再現できる.
\begin{flushright}
    \cite[p.144]{認知心理学辞典}
\end{flushright}
\section{固有顔法}
固有顔(eigenface)とは,顔画像の認識において最も有名な手法のひとつ.
顔認識では,顔を構成する部品(目や鼻,口など)の形状や,配置から特徴点を抽出して認識に利用する.しかし,照明方向や,撮影距離,表情,顔の傾きなどで顔の見え方が変わる.これらを許容したうえで正しく顔認識するには,顔画像をパターンとして扱い,統計的パターン認識手法を適用する方法がある.

まず挙げられるのが,パターン間のマッチングを用いた方法である.この方法は最も簡単なパターン認識手法であるが,画像そのものをパターンとして扱った場合,パターンの次元が大きくなる.
この問題を解決する方法として提案されたのが,固有顔である.固有顔は主成分分析によりパターンを情報圧縮し,顔画像の識別に利用している.

数枚の画像に対して,各画像から平均ベクトルを引いた集合に対して,固有値を求める.この時の固有ベクトルを固有顔と呼ぶ.\par
\hfill\cite{顔画像からの個人識別}
\section{オプティカルフロー}
\section{ストラクチャフロムモーション(SfM: Structure from Motion)}
