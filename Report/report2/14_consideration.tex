\section{\consideration}
\paragraph{\kadaiaa}
実験結果より,原画像に近い画像は,緑チャネルだけを抜き出してグレイスケール画像として表示したものである.
また,赤チャネルと青チャネルを入れ替えた画像では,文字や建物の概形を,はっきりとらえることができる.
これらは,\eqref{equ:grayscale}でGreenの割合が一番多い理由と考えられる.
\paragraph{量子化数変換・階調変換・閾値処理}
量子化数を減らすことで,等高線のようなものが見える.
これは,元画像でなだらかな画素値の変化箇所が,量子化数を減らすことでとびとびの値になることで生じる.これを擬似輪郭という.
擬似輪郭は,量子化数を減らすことでより顕著になる.
量子化数が減ると擬似輪郭が生じる.この擬似輪郭は,階調反転しても変わらない.
さらに,閾値を調節することで,擬似輪郭を境に白と黒に別れることがわかった.
\paragraph{Sobelフィルタ,メディアンフィルタ}
メディアンフィルタは,ノイズに強いことがわかる.これは中央値を算出することにより,ノイズでない値が中央値となるからである.
ただし,画素値の中央値を抽出するため,平均値を算出する平滑化フィルタに比べて,計算量が多い.
それに対して,平滑化フィルタは,ノイズの画素値を含めた値の平均を算出するので,メディアンフィルタに比べるとノイズに弱い.
\paragraph{Sobelフィルタ,Laplacianフィルタ}
Sobelフィルタは,画像の隣接画素間の差分を計算する.よって,差が激しい画素の画素値が大きくなり,白く表示される.
つまり,ノイズに対してもエッジが強調される.
Laplacianフィルタは,画像の二次微分を計算することでエッジの位置を抽出している.Laplacianフィルタは高周波成分を増幅するため,ノイズに対してもエッジが強調される.
\paragraph{背景差分画像・色空間変換}
背景差分画像では,被写体がぼんやりと写っている.この画像に対して閾値処理することで,被写体を強調できることが分かった.
今回の実験では,閾値を手探りで探したため,実験結果の閾値が最適か分からない.
また,実験結果より,強調された部分は,光の当たり方や影の変化で白く光る部分もある.
ゆえに,背景差分画像だけでは,被写体の輪郭を正確にとらえることができないと考えられる.
HSV色空間では,この問題を解決できる.
RGB色空間では,明暗(影や光の状況)でRGB値が変わるのに対して,HSV色空間では,明暗が「明度」というチャネルで保存されているので,光の状況や影に左右されず,肌色領域をより正確に抽出できる.
