\section{\consideration}
\paragraph{\kadaiaa}
実験結果より,原画像に近い画像は,緑チャネルだけを抜き出してグレイスケール画像として表示したものである.
また,赤チャネルと青チャネルを入れ替えた画像では,文字や建物の概形を,はっきりとらえることができる.
これらは,\eqref{equ:grayscale}でGreenの割合が一番多い理由と考えられる.
\paragraph{量子化数変換・階調変換・閾値処理}
量子化数を減らすことで,等高線のようなものが見える.
これは,元画像でなだらかな画素値の変化箇所が,量子化数を減らすことでとびとびの値になることで生じる.これを擬似輪郭という.
擬似輪郭は,量子化数を減らすことでより顕著になる.
量子化数が減ると擬似輪郭が生じる.この擬似輪郭は,階調反転しても変わらない.
さらに,閾値を調節することで,擬似輪郭を境に白と黒に別れることがわかった.
\paragraph{画像処理フィルタ}
メディアンフィルタは,ノイズに強いことが分かる.これは中央値を算出することにより,ノイズでない値が中央値となるからである.
ただし,画素値の中央値を抽出するため,平均値を算出する平滑化フィルタに比べて,計算量が多い.
それに対して,平滑化フィルタは,ノイズの画素値を含めた値の平均を算出するので,メディアンフィルタに比べるとノイズに弱い.
Sobelフィルタは,画像の隣接画素間の差分を計算する.よって,差が激しい画素の画素値が大きくなり,白く表示される.
つまり,ノイズに対してもエッジが強調される.
Laplacianフィルタは,画像の二次微分を計算することでエッジの位置を抽出している.Laplacianフィルタは高周波成分を増幅するため,ノイズに対してもエッジが強調される.
\paragraph{背景差分画像・色空間変換}
背景差分画像では,被写体がぼんやりと写っている.この画像に対して閾値処理することで,被写体を強調できることが分かった.
今回の実験では,閾値を手探りで探したため,実験結果の閾値が最適か分からない.
また,実験結果より,強調された部分は,光の当たり方や影の変化で白く光る部分もある.
ゆえに,背景差分画像だけでは,被写体の輪郭を正確にとらえることができないと考えられる.
HSV色空間では,この問題を解決できる.
RGB色空間では,明暗(影や光の状況)でRGB値が変わるのに対して,HSV色空間では,明暗が「明度」というチャネルで保存されているので,光の状況や影に左右されず,肌色領域をより正確に抽出できる.
\begin{wrapfigure}{r}[0mm]{.3\textwidth}
    \centering
    \begin{tikzpicture}
        \draw[latex-latex](-0.5,1.5)node[left]{\rotatebox{90}{\tiny High}}--(3.5,1.5)node[right]{\rotatebox{90}{\tiny High}};
        \draw[latex-latex](1.5,-0.5)node[below]{\tiny High}--(1.5,3.5)node[above]{\tiny High};
        \draw (0,0)rectangle(3,3);
        \coordinate (O) at (1.5,1.5);
        \node[below right] at (O) {\tiny Low};
        \draw[Stealth-Stealth](4,0)--(4,3)node[midway,fill=white]{\rotatebox{90}{\tiny \(y\)方向スペクトル}};
        \draw[Stealth-Stealth](0,-1)--(3,-1)node[midway,fill=white]{\tiny \(x\)方向スペクトル};
    \end{tikzpicture}
    \caption{パワースペクトル}
    \label{fig:パワースペクトル}
    \vspace{-1cm}
\end{wrapfigure}
\paragraph{2次元フーリエ変換}
2次元フーリエ変換により得られるパワースペクトルは,中央に低周波数,周辺に高周波成分が分布する.
パワースペクトルの輝度は,含まれてる周波数成分の量を示す.
正弦波縞パワースペクトルの結果より,\figref{fig:パワースペクトル}の結果と一致している.
縦縞は,横方向に明暗が存在し,縦方向の輝度は変わらないので,パワースペクトルは,横方向に存在し,逆もまた然りである.
さらに,縞の数が多い,つまり周波数が高いならば,パワースペクトルは外側が明るくなることも,実験結果と一致する.
一方,実験結果より,画像座標とパワースペクトル座標は,依存関係にないことがわかる.
同じ大きさの長方形の位置によってパワースペクトルは,完全一致はしないものの,大きく異ならないことが実験結果からわかる.
\paragraph{高域通過フィルタ}
低周波領域にフィルタをかけて,高周波領域のみを通過させるフィルタを適用すると,縦横方向のエッジの間隔が狭い領域を確認できた.
具体的には,髪の毛や目,帽子や口などを確認できた.
円形フィルタの半径を大きくすると,さらにエッジの間隔が狭い部分のみ確認できた.
具体的に,フィルタ半径\(50\textrm{pixel}\)の画像で確認できた帽子や口は,フィルタ半径\(100\textrm{pixel}\)の画像では確認できなかった.
