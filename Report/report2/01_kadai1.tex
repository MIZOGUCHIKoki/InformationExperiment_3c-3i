\chapter{\kadaia}
\section{\purpose}
本章では,画像のカラーチャネル操作,量子化,階調反転,閾値処理を行う.またグレイスケール画像に対してヒストグラムを作成する.
\paragraph{\kadaiaa}RGB色空間の画像を,緑チャネルだけを抜き出してグレイスケール画像を作成する.緑チャネルは色の濃淡を多く含む.RGB色空間から色の濃淡を抽出したい場合は,緑(G)の成分を多く抽出するとよい.具体的な割合を,\eqref{equ:grayscale}に示す.ここではNTSC輝度信号を取り出す方法で行う.
さらに,RGB画像の赤チャネルと青チャネルを入れ替えたカラー画像を作成する.
\begin{align}
    \textrm{Gray scale image} & = \textrm{Red}\times 30\% +\textrm{Green}\times 59\% +\textrm{Blue}\times 11\%\label{equ:grayscale}
\end{align}
\paragraph{\kadaiab}画像の量子化ビット数を変更することによる,画像の変化を確認する.量子化数は8Bit,4Bit,2Bit,1Bitの4種をテストする.量子化数1Bitの画像を2値画像という.
\paragraph{\kadaiac}各量子化数の画像に対して,その画像を階調反転させる.階調反転とは,白黒を反転させることである.量子化数による階調変換後の画像を比較する.
\paragraph{\kadaiad}閾値処理とは,とある値(閾値)以上の場合を白,閾値以下を黒とし,2値画像を作成することである.
\paragraph{\kadaiae}量子化数8Bitのグレイスケール画像のヒストグラムを作成する.画素値\(n(n=0,1,\dots ,255)\)の画素が何画素含まれているかのヒストグラムを作成する.
\paragraph{\kadaiaf}自分が写っている写真と,背景だけが写っている写真の差分画像をとる.これを背景差分と呼ぶ.背景差分の後,閾値処理を行う.物体領域を正しく検出するために考慮する点を考察する.