\chapter{\kadaia}
\section{\purpose}
本章では,画像のカラーチャネル操作,量子化,階調反転,閾値処理を行う.またグレイスケール画像に対してヒストグラムを作成する.
\paragraph{\kadaiaa}RGB色空間の画像を,緑チャネルだけを抜き出してグレイスケール画像を作成する.緑チャネルは色の濃淡を多く含む.RGB色空間から色の濃淡を抽出したい場合は,緑(G)の成分を多く抽出するとよい.具体的な割合を,\eqref{equ:grayscale}に示す.ここではNTSC輝度信号を取り出す方法で行う.
さらに,RGB画像の赤チャネルと青チャネルを入れ替えたカラー画像を作成する.
\begin{align}
    \textrm{Gray scale image} & = \textrm{Red}\times 30\% +\textrm{Green}\times 59\% +\textrm{Blue}\times 11\%\label{equ:grayscale}
\end{align}
\paragraph{\kadaiab}画像の量子化数を変更することによる,画像の変化を確認する.量子化数は8Bit,4Bit,2Bit,1Bitの4種をテストする.量子化数1Bitの画像を2値画像という.
\paragraph{\kadaiac}各量子化数の画像に対して,その画像を階調反転させる.階調反転とは,白黒を反転させることである.量子化数による階調変換後の画像を比較する.
\paragraph{\kadaiad}閾値処理とは,とある値(閾値)以上の場合を白,閾値以下を黒とし,2値画像を作成することである.
\paragraph{\kadaiae}量子化数8Bitのグレイスケール画像のヒストグラムを作成する.画素値\(n(n=0,1,\dots ,255)\)の画素が何画素含まれているかのヒストグラムを作成する.
\paragraph{\kadaiaf}自分が写っている写真と,背景だけが写っている写真の差分画像をとる.これを背景差分と呼ぶ.背景差分の後,閾値処理を行う.物体領域を正しく検出するために考慮する点を考察する.
\section{\method}
\paragraph{実験に用いる装置}このレポート内すべての実験にはMathWorks\raisebox{2mm}{\tiny\textregistered}社の\matlab を用いて,\tblref{tbl:実験環境}の環境下で実験する.
\begin{table}[H]
    \caption{実験環境}
    \label{tbl:実験環境}
    \begin{tabularx}{\textwidth}{AR}
        \hline
        実験機                      & MacBook Air 2022 (Apple社)\texttt{MLY13J/A}    \\
        プロセッサ                    & Apple Silicon M2\ \  8コアCPU,8コアGPU            \\
        メモリ                      & 8GB                                           \\
        \multirow{2}{*}{\matlab} & R2023a - academic use (Update1 9.14.02239454) \\
                                 & 64-bit (maci64) March 30, 2023                \\
        \hline
    \end{tabularx}
\end{table}
また,このレポートないすべての実験では\matlab でプロットしたグラフを出力するための\texttt{exportgraphics}関数,画像を書き出すための\texttt{imwrite}関数を用いる(\srcref{src:グラフ・画像出力}).
\begin{lstlisting}[numbers={none},caption={グラフ・画像出力},label={src:グラフ・画像出力}]
exportgraphics(figurename,'path/figure_name.pdf','ContentType','vecto');
imwrite(data,'path/figure_name.png');
\end{lstlisting}
\paragraph{\kadaiaa}
\texttt{imwrite}関数を用いて,画像の読み込む.読み込んだ画像はRGB色空間で保存されており,チャネル1にはR,チャネル2にはG,チャネル3にはBが保存されている.
グレイスケール画像を作成するには,\eqref{equ:grayscale}の割合で画像を加算合成する.\mat{m}{n}の行列\texttt{A}に対して,\mat{1}{n}を取り出したければ,\verb|A(1,:)|と記述する.\verb|:|は,すべての要素を表す記号である.
赤チャネルと青チャネルを入れ替えるためには,赤チャネルの行列と青チャネルの行列を変数に保存し,それぞれ互いのチャネルに代入する.\scall{\kadaiaa}\sref{src:01_01}.

\begin{wrapfigure}{r}[0mm]{.3\textwidth}
    \centering
    \vspace{-.5cm}
    \begin{lstlisting}[caption={\texttt{bitshift}関数},label={src:bitshift}]
img = bitshift(img, n);
    \end{lstlisting}
    \vspace{1cm}
    \begin{tikzpicture}
        \coordinate (A) at (4,1);
        \draw (0,0.5)--(4,0.5);
        \draw (0,0)rectangle(A);
        \foreach \x in {1,2,...,8}
            {
                \draw (\x/2,0.4)--(\x/2,0.6);
                \draw (\x/2,1)--(\x/2,0.9);
                \draw (\x/2,0)--(\x/2,0.1);
                \node at ($(\x/2-1/2,0.5)!0.5!(\x/2,1)$){\texttt{1}};
                \node at ($(\x/2-1/2,0)!0.5!(\x/2,0.5)$){\ifnum\x<5\texttt{0}\else\texttt{1}\fi};
            }
    \end{tikzpicture}
    \caption{4ビットシフト}
    \label{fig:4ビットシフト}
    \vspace{-.5cm}
\end{wrapfigure}
\paragraph{\kadaiab}
量子化数を変更するために,\texttt{bitshift}関数を用いる(\srcref{src:bitshift}).
この関数は,\texttt{img}を左に\texttt{n}ビットシフトする関数である.右シフトしたい場合は\texttt{n}を負の数にすれば良い.
ビットシフトについて,1ビット右シフトするごとにそのデータは\(1/2\)される.
これを利用して,量子化数4Bitの場合は右に4Bitシフト,量子化数2Bitの場合は右に6Bitシフト,量子化数1Bitの場合は右に7Bitシフトすると良い.
量子化数4Bitを例にあげる.右に4Bitシフトするとき,
\paragraph{\kadaiac}各量子化数ごとに階調反転する.