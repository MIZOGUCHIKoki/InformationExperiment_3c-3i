\section{\method}
\paragraph{実験に用いる装置}このレポート内すべての実験にはMathWorks\raisebox{2mm}{\tiny\textregistered}社の\matlab を用いて,\tblref{tbl:実験環境}の環境下で実験する.
\begin{table}[H]
    \caption{実験環境}
    \label{tbl:実験環境}
    \begin{tabularx}{\textwidth}{AR}
        \hline
        実験機                      & MacBook Air 2022 (Apple社)\texttt{MLY13J/A}    \\
        プロセッサ                    & Apple Silicon M2\ \  8コアCPU,8コアGPU            \\
        メモリ                      & 8GB                                           \\
        \multirow{2}{*}{\matlab} & R2023a - academic use (Update1 9.14.02239454) \\
                                 & 64-bit (maci64) March 30, 2023                \\
        \hline
    \end{tabularx}
\end{table}
また,このレポートないすべての実験では\matlab でプロットしたグラフを出力するための\texttt{exportgraphics}関数,画像を書き出すための\texttt{imwrite}関数を用いる(\srcref{src:グラフ・画像出力}).
\begin{lstlisting}[numbers={none},caption={グラフ・画像出力},label={src:グラフ・画像出力}]
exportgraphics(figurename,'path/figure_name.pdf','ContentType','vector');
imwrite(data,"path/figure_name.png");
\end{lstlisting}
\paragraph{\kadaiaa}
\texttt{imwrite}関数を用いて,画像の読み込む.読み込んだ画像はRGB色空間で保存されており,チャネル1にはR,チャネル2にはG,チャネル3にはBが保存されている.
グレイスケール画像を作成するには,\eqref{equ:grayscale}の割合で画像を加算合成する.\mat{m}{n}の行列\texttt{A}に対して,\mat{1}{n}を取り出したければ,\verb|A(1,:)|と記述する.\verb|:|は,すべての要素を表す記号である.
赤チャネルと青チャネルを入れ替えるためには,赤チャネルの行列と青チャネルの行列を変数に保存し,それぞれ互いのチャネルに代入する.\scall{\kadaiaa}\sref{src:05_01}.

\begin{wrapfigure}{r}[0mm]{.3\textwidth}
    \centering
    \vspace{-.7cm}
    \begin{lstlisting}[caption={\texttt{bitshift}関数},label={src:bitshift}]
img = bitshift(img, n);
    \end{lstlisting}
    \vspace{-.5cm}
\end{wrapfigure}
\paragraph{\kadaiab}
量子化数を変更するために,\texttt{bitshift}関数を用いる(\srcref{src:bitshift}).
この関数は,\texttt{img}を左に\texttt{n}ビットシフトする関数である.右シフトしたい場合は\texttt{n}を負の数で与える.
ビットシフトについて,1ビット右シフトするごとにそのデータは\(1/2\)される.
これを利用して,量子化数4Bitの場合は右に4Bitシフト,量子化数2Bitの場合は右に6Bitシフト,量子化数1Bitの場合は右に7Bitシフトする.
量子化数4Bitを例にあげる.仮に画素値が\texttt{255}(白)を持つ画素の場合,量子化数を4Bitにする,つまり4Bit右シフトすると,画素値は\texttt{15}になる.このままでは画素値の範囲が\(0\)から\(15\)となる.
この対策として,全体画素値と\(255/15\)の積を取ることで,画素値を\(0\)から\(255\)にスケーリングする.\scall{\kadaiab}\sref{src:05_02}.
\paragraph{\kadaiac}
各量子化数ごとに階調反転する.階調反転を実現するためには,階調反転した画像を\texttt{double}型に変換したあと,\(-1\)との積をとり,\(255\)を足した後で\texttt{unit8}型に変換する\footnote{その画像の各画素値が\texttt{double}型であるとき,\texttt{imwrite}が,データを自動的にリスケールし書き出すため.}.\scall{\kadaiac}\sref{src:05_03}.

\begin{wrapfigure}{r}[0mm]{.3\textwidth}
    \centering
    \vspace{-.7cm}
    \begin{lstlisting}[caption={判定結果の格納},label={src:判定結果の格納}]
mat = [1 2 3; 4 5 6; 7 8 9];
bin = mat > 5;
% -- 結果 --
bin = [0 0 0; 0 0 1; 1 1 1];     
    \end{lstlisting}
    \vspace{-.5cm}
\end{wrapfigure}
\paragraph{\kadaiad}
\matlab には,判定結果のブール値を行列に格納する機能がある.
\srcref{src:判定結果の格納}より,行列\texttt{mat}の各元が\(5\)より大きい箇所を\(1\),\(5\)以下のところを\(0\)とする行列\texttt{bin}を作成できる.この行列を真理行列と呼ぶ.
この機能を用いて,ある閾値に対して,閾値よりも大きければ\(1\)を戻し,閾値以下であれば\(0\)を戻す行列を作成する.画素値の範囲を\(0\)から\(255\)へするために,行列の各元と\(255\)の積をとる.今回は,閾値を\(64\),\(128\),\(192\)で実験する.\scall{\kadaiad}\sref{src:05_04}

\begin{wrapfigure}{r}[0mm]{.3\textwidth}
    \centering
    \vspace{-.9cm}
    \begin{lstlisting}[caption={\texttt{sum}関数},label={src:sum関数}]
matA = [1 2 3];
s_matA = sum(matA); 
% -> 出力:6
matB = [1 2; 1 1; 1 1];
s_matB = sum(matB); 
% -> 出力:[3 4]
s_s_matB = sum(sum(matB)); 
% -> 出力:12
    \end{lstlisting}
    \vspace{-.7cm}
\end{wrapfigure}
\paragraph{\kadaiae}
ヒストグラムを作成するために,この関数は行列の元を足し合わせる\texttt{sum}関数を用いる(\srcref{src:sum関数}).
各画素値\(0\)から\(255\)に対して,その画素値と等しい箇所を\(1\)とする真理行列を作成し,各元の和を\texttt{sum}関数を用いて算出する.
その結果が,ある画素値がいくつ画像に含まれているかを指す.
\paragraph{\kadaiaf}
固定カメラ\footnote{手での固定は,背景がズレる可能性があるので,カメラを固定して撮影した.}で撮影した写真を用いる.「背景と被写体が写っている画像\texttt{img\_sbj}」「背景のみの画像\texttt{img\_bg}」の2点を撮影した.背景差分画像は,\(\texttt{img\_sbj}-\texttt{img\_bg}\)で生成する.
生成した画像に対して,閾値処理する.閾値処理する前後での画像比較,閾値による比較し,考察する.今回,閾値を\(32\),\(64\),\(128\)で実験する.\scall{\kadaiaf}\sref{src:05_06}.

\begin{wrapfigure}{r}[0mm]{.3\textwidth}
    \vspace{-.5cm}
    \begin{lstlisting}[caption={白色ガウス雑音画像の生成},label={src:白色ガウス雑音画像の生成}]
% 画像サイズ : n x m
wgn = 10*randn(n, m);
wgn = uint8(wgn);
wgn_img = wgn + gimg;
    \end{lstlisting}
    \begin{lstlisting}[caption={インパルス雑音画像の生成},label={src:インパルス雑音画像の生成}]
% 画像サイズ : n x m
rnd = rand(n, m);
b = (rnd < 0.01);
w = (rnd > 0.99);
in_img(w) = 255;
in_img(b) = 0;
    \end{lstlisting}
    \vspace{-1cm}
\end{wrapfigure}
\paragraph{白色ガウス雑音を加えた画像}白色ガウス雑音の作成には\texttt{randn}関数を用い,生成した乱数と,標準偏差の積を取る.生成した乱数を\texttt{uint8}型に変換し,\originimg と和を取る(\srcref{src:白色ガウス雑音画像の生成}).
\(255\)を上回る,または\(0\)を下回る場合,それぞれの値に変換したものを, \wgnimg とする.
\paragraph{インパルス雑音を加えた画像}
インパルス雑音は,\texttt{rand}関数を用いて作成する.発生率を\(1\%\) にするため,乱数の\(0.01\)未満の画素を黒,乱数の\(0.99\)より大きい画素を白としてインパルスノイズを設計する(\srcref{src:インパルス雑音画像の生成}).
\paragraph{フィルタの適用}
平滑化フィルタ,微分フィルタ,Laplacianフィルタの適用は,\texttt{filter2}関数,(\ \verb|filter2(filter, img)|\ )を用いる.
メディアンフィルタは,画像の\mat{i}{j}に対して,\texttt{median}関数を用いてフィルタ処理する.メディアンフィルタを適用する際に,四方すべてに画素がない画素(\mat{1}{1}画素,\mat{m}{1}画素など)はフィルタ処理できないため,0パディング処理を行い,メディアンフィルタを適用する(\srcref{src:メディアンフィルタの適用}).
\begin{lstlisting}[caption={メディアンフィルタの適用},label={src:メディアンフィルタの適用},frame={left}]
for h = 2:img_height % 画像行列 img の高さ
    for w = 2:img_width % 画像行列 img の横幅,median_filter は0パディング後のimg
        median_filter(h-1, w-1) = median(img(h-1:h+1,w-1:w+1),"all"); 
    end
end
\end{lstlisting}
課題(フィルタ処理)のスクリプトは,\sref{src:06_01} - \sref{src:06_04}.
\paragraph{色空間変換}
読み込んだ画像はRGB色空間で保存される.この画像をHSV色空間に変換するためには,\texttt{rgb2hsv}関数を用いる.
出力された値と\(255\)の積を取り,HSV色空間で出力された画像を書き出す.
色相,彩度,明度それぞれのチャネルを抽出し,\matlab のアプリケーションを用いて,肌色要素のHSV成分を出力する(\sref{src:06_05_f}).
それぞれの値に合致した画素を,画素値\(255\),ほかの画素値を\(0\)とした画像を書き出す.同様な方法で,RGB色空間における肌色領域の抽出も行う(\sref{src:06_05_f2}).
\scall{\kadaibe}\sref{src:06_05},\sref{src:06_05_1}.
