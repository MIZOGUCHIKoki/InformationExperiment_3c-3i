\chapter{\kadaic}
\section{\method}
\paragraph{\kadaica}
\matlab で,方位を定義できる正弦波縞を作成する.
方位\(\theta\),空間周波数\(f\),コントラスト\(C\)の正弦波縞は\eqref{equ:正弦波縞}で表せる.
\(L(x,y)\)を点\(L(x,y)\)における輝度,\(L_0\)を全体画像の平均輝度とする.
\begin{align}
    L(x,y) & = L_0\Bigg(1+C\sin\Big(2\pi f\big(y\sin(\theta)+x\cos(\theta)\big)\Big)\Bigg)\label{equ:正弦波縞}
\end{align}
左に\(30^\circ\)傾いた空間周波数\(0.05\textrm{cycle}/\textrm{pixel}\)の正弦波縞と,
右に\(60^{\circ}\)傾いた空間周波数\(0.03\textrm{cycle}/\textrm{pixel}\)の正弦波を作成する.
平均輝度は,最大輝度の\(0.5\)倍,コントラストを\(0.5\)とする.
\paragraph{\kadaicb}
方位\(90\pm 10^\circ\),\(90\pm 45^\circ\)正弦波縞の順応刺激を上下に提示して,\(60\)秒後に垂直縞\(90^\circ\)のテスト刺激を表示する.
全体画像サイズを\(\textrm{縦}900\times\textrm{横}400\textrm{pixel}\),格子縞のサイズを\(400\times 400\textrm{pixel}\),順応時に提示する中央の矩形を\(20\times 100\textrm{pixel}\),順応後の注視点を\(20\textrm{pixel}\)とする.
正弦波縞の空間周波数を\(0.03\textrm{cycle}/\textrm{pixel}\),コントラストは\(0.5\)とする.
