\chapter{画像情報処理}
\section{\purpose}
\matlab を用いて,画像に対して,カラーチャネルの操作,量子化数変換,階調反転,閾値処理,画素値に対するヒストグラムを作成など,基本的な画像処理を行う.\par
また,HSV色空間上での肌色抽出する.HSV色空間の特徴として,人間が色を知覚する方法と類似しており,視覚障害者向けのアクセシビリティ向上に役立つことが挙げられる\cite[p.97\ -\ p.98]{画像処理}.
人間が色を知覚する方法と類似していることを踏まえて,HSV色空間を用いることで,画像の特徴を抽出しやすくなる.今回の実験では,自分の手の写真をRGB色空間からHSV色空間へ変換し,肌色領域を抽出する.抽出した肌色領域を白色,そのほかの部分を黒色にして出力する.出力した画像と,RGB色空間における肌色領域を抽出した場合の精度について考察する.\par
さらに,ノイズ画像に対して,ノイズ除去する.周囲画素値の平均値を中央の画素値にする平滑化フィルタと,周囲画素値の中央値を中央の画素値にするメディアンフィルタを,ノイズの雑音除去具合で比較する.\par
加えて,画像のエッジを検出するために,微分フィルタを適用する.今回適用する微分フィルタは,SobelフィルタとLaplacianフィルタである.両フィルタで抽出できるエッジの精度について考察する.\par
最後に,画像を2次元フーリエ変換し,パワースペクトルを出力する.出力したパワースペクトルに対して,高域通過フィルタを適用し,画像座標とパワースペクトル画像座標の関係と,パワースペクトル各値の意味を明らかにする.