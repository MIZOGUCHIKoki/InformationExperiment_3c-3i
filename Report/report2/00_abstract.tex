\renewcommand{\abstractname}{概要}
\begin{abstract}
    本実験では,画像情報処理,視覚情報処理を扱う.画像情報処理では,カラーチャネル操作や閾値処理,量子化数変換などの基礎的な画像処理を行い,それらの技術を用いて画像フィルタや背景差分画像の作成や肌色領域の抽出や,画像の2次元フーリエ変換をする.パワースペクトルの示す意味や,画像と,画像のパワースペクトルの座標の関係が明らとなった.また,背景差分画像と色空間変換後の肌色領域抽出で,対象物の抽出における特徴も明らかとなった.
    視覚情報処理では,方位残効を扱う.\matlab 上で方位残効刺激を作成する.この実験より,順応刺激とする縞模様方位の大きさと方位残効の度合いの関係を,視覚特性を考慮したうえで明らかになった.
\end{abstract}