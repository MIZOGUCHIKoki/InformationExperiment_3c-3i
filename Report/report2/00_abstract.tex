\renewcommand{\abstractname}{概要}
\begin{abstract}
    本実験では,画像情報処理,視覚情報処理を扱う.画像情報処理では,カラーチャネル操作や閾値処理,量子化数変換などの基礎的な画像処理を行い,それらの技術を用いて画像フィルタや背景差分画像の作成や肌色領域の抽出,2次元フーリエ変換する.2次元フーリエ変換のパワースペクトルと原画像の関係や,2次元フーリエ変換後の出力と,画像座標との関係が明らかになった.また,背景差分画像と色空間変換後の肌色領域抽出で,対象物の抽出における特徴も明らかとなった.
    視覚情報処理では,方位残効を扱う.\matlab 上で方位残効刺激を作成する.この実験より,縞模様方位の大きさと,方位残効の度合いの関係を,ヒトの視覚情報処理を考慮したうえで明らかになった.
\end{abstract}